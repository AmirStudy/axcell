\documentclass{article}
\usepackage{booktabs}
\usepackage{xcolor}
\title{DILBERT: Distilling Inner Latent BERT variables}
\author{John Doe}
\begin{document}
\maketitle
\begin{abstract}
  In this paper we achieve state-of-the-art performance in random number generation.
\end{abstract}
\section{Introduction}
\section{Model}
\subsection{Preprocessing}
\subsection{Architecture}
\section{Experiments}
In this section we present Table~\ref{tab}.
  \begin{table}
    \begin{tabular}{lcr} \toprule
      left & center & right\\\midrule
      1 & 2 & 3\\
      4 & 5 & 6\\\midrule
      7 & 8 & 9\\
      a & b & c\\\bottomrule
    \end{tabular}
    \caption{A table.}
    \label{tab}
  \end{table}
  \begin{table}
    \begin{tabular}{lcr} \toprule
      \textbf{bold text} & \textit{italic text} & \textbf{\textit{bold italic text}}\\\midrule
      \textcolor{red}{red text} & \textcolor{green}{green text} & \textcolor{blue}{blue text}\\
      $\mathbf{5.4\%}$ & $\mathit{3.8\%}$ & $\mathbf{11.2}\pm 0.15$\\\midrule
      \textbf{an \textit{italic} text inside bold}  & {\bf \textcolor{red}{bold red}} & \\\bottomrule
    \end{tabular}
    \caption{A table.}
    \label{tab2}
  \end{table}
\end{document}
